\documentclass{article}
\usepackage{cite}
\usepackage[colorlinks=true,linkcolor=black,urlcolor=black]{hyperref}
\renewcommand{\t}{\texttt}

\title{15-744 Project Proposal}
\author{Carlo Angiuli \and Michael Sullivan}
\date{September 28, 2011}

\begin{document}
\maketitle

% Needs to include:
%  - problem statement
%  - state-of-the-art
%  - work plan / milestones

\section{Motivation}

Modern network applications need to support many simultaneous clients---this is
known as the C10K (``10,000 client'') problem\cite{Kegel}. But to maintain these
simultaneous connections, an application must track each client's state
independently. A straightforward way to do this is to spawn a thread for each
client, but this approach does not scale as the number of clients grows. 

A lightweight approach is to use kernel mechanisms for multiplexed, non-blocking
I/O, such as \t{select}, \t{poll}, \t{epoll}, or \t{kquery}; unfortunately, to
maintain separate state for each client, application programmers must manually
structure their code as state machines, which is tedious and error-prone.
Unsurprisingly, there exist many thin C/C++ wrappers for these facilities, but
none avoid major code refactoring.

To maintain the illusion of threads (in particular, client-specific state),
languages like Erlang and Haskell provide lightweight threading via user-space
schedulers and non-blocking IO. While this affords better performance, the full
generality of this approach still imposes unnecessary overhead on networking
applications which don't require preemptable, independently scheduled threads
\cite{Vinoski}.

\section{Proposal}

Hence, we intend to produce an embedded domain-specific language in Haskell to
describe concurrent network applications. This language will provide composable,
high-level network operations, in which programmers may express their
application logic naturally, as if using threads. 

However, this high-level description will be automatically synthesized into a
state machine representation usable in a C loop invoking \t{epoll}, using no
threads whatsoever. We will additionally output a graphical representation of
this underlying state machine.

The back-end of this system can be implemented via several different options
which we are still evaluating. One option is to directly interface with the
\t{epoll} hooks in the Haskell runtime system, bypassing the native lightweight
threads. Another option is to generate C source code which itself implements the
network application, optionally linkable with native C code.

\section{Milestones}

We will first collect a representative sample of Linux network applications and
examine the range of high-level primitives required to support them. We will
then decide on a specific back-end for our language, and design the front- and
back-end in tandem. To evaluate the utility of this language, we will build a
number of representative applications in our language (e.g., an echo server,
simple HTTP server) and compare them to traditional implementations, with
respect to both code simplicity and performance.

\bibliography{citations}{}
\bibliographystyle{plain}

\end{document}

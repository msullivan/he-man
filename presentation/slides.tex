\documentclass{beamer}
\usepackage{graphics}
\renewcommand{\t}{\texttt}

% Define Beamer appearance
\usetheme{Montpellier}
\setbeamertemplate{navigation symbols}{}
\setbeamertemplate{headline}{} % comment out if you want navtree on slides
\setbeamertemplate{footline}[frame number]

\title{He-Man}
\subtitle{Haskell Event Manager, Apropos Networking}
\author{Carlo Angiuli \and Michael Sullivan}
\date{December 12, 2011}

\begin{document}

\begin{frame}
\maketitle
\end{frame}

\begin{frame}
\frametitle{The C10K problem}
How do you write network applications that scale to 10k clients?
\begin{itemize}
\item Concurrency is essential.
\item Must maintain per-client state.
\end{itemize}
\end{frame}

\begin{frame}
\frametitle{Solutions}
\begin{itemize}
\item Threads: Write code in the obvious, straight line manner, using blocking
IO. Memory-intensive, scheduled by OS.
\item Event loops: Use non-blocking IO and multiplexing like \t{epoll()}.
Requires refactoring as a state machine.
\item Threads are easier to program, event loops perform better.
\end{itemize}
\end{frame}

\begin{frame}[fragile]
\frametitle{Threads}
\small
\begin{verbatim}
main() {
  ...
  while (1) {
    fd = accept(...);
    spawn(thread, fd);
  }
}

thread(fd) {
  while (1) {
    len = read(fd, buf, SIZE);
    written = 0;
    while (written < len)
      written += write(fd, buf + written, len - written);
  }
}
\end{verbatim}
\end{frame}

\begin{frame}
\frametitle{Event loops}
\includegraphics[scale=.5]{graph.pdf}
\end{frame}

\begin{frame}
\frametitle{Best of both worlds}
Write straight-line code and generate state machine automatically.
\begin{itemize}
\item Embedded domain-specific language in Haskell.
\item Compiles to a C event loop, links to a small runtime \\
(and the rest of your code).
\item Generates basic blocks not containing blocking IO.
\item All blocking occurs in the runtime's main loop.
\end{itemize}
\end{frame}

\begin{frame}
\frametitle{Status}
\begin{itemize}
\item We have a working echo server and rudimentary HTTP server.
\item Supports asynchronous disk IO and non-blocking network IO.
\item Doing performance testing on HTTP server.
\end{itemize}
\end{frame}

\begin{frame}
\frametitle{Future work}
\begin{itemize}
\item Multithreaded event loop.
\item Typed source language (via phantom types or GADTs).
\item Optimized runtime.
\end{itemize}
\end{frame}

\end{document}

\end{document}

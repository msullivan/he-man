\documentclass{article}
\usepackage{cite}
\renewcommand{\t}{\texttt}

\title{15-744 Project Proposal}
\author{Carlo Angiuli \and Michael Sullivan}
\date{September 28, 2011}

\begin{document}
\maketitle

% Needs to include:
%  - problem statement
%  - state-of-the-art
%  - work plan / milestones

\section{Motivation}

It's good for network applications to be able to support many simultaneous
connections---this is known as the C10K (``10,000 client'') problem\cite{c10k}. 

Simultaneous client connections require an application to maintain multiple
states at once. A classic way to do this is to spawn a thread for each client.
However, this approach does not scale as the number of clients grows. 

(nonblocking?)
(lightweight threads?)
(cite Erlang)

A lightweight method is to use \t{epoll} or \t{kquery}.
This requires the application programmer to manually maintain a state machine
with 

Requires a lot of complex but straightforward code which is difficult to debug.
Unsurprisingly, there exist many thin C/C++ wrappers, but these still require
...

and provide no assurance that the program ...

\section{Proposal}

We present a domain-specific language which provides a simple

easily extensible with more protocol definitions

outputs C code to be linked with the rest of the server code.

formal guarantees: given our abstract specification of \t{epoll}, our DSL
generates code with the following properties...

embedded in Haskell: natively allows functional metaprogramming,
leverages native type system

\section{Milestones}



\bibliography{citations}{}
\bibliographystyle{plain}

\end{document}
